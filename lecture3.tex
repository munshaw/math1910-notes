\chapter{Set Theory 3}

Sets can have a finite or infinite number of elements. Beleive it or not, we can still determine wether or not two infinite sets have the same cardinality using the following definition.

\begin{boxdefine}{One-to-one Correspondence}{}
	A {\bf one-to-one correspondence} between sets $A$ and $B$ is a pairing of each object in $A$ with one and only one object in $B$, and each object in $B$ paired with one and only one object in $A$. If such a correspondence exists, than we say $A$ and $B$ are {\bf equivalent} (that is $|A|=|B|$).
\end{boxdefine}

\begin{boxexample}{}{}
	If $A=\{1,2,3,4,\dots\}$ and $B=\{1,4,9,16,\dots\}$, we can say that the one-to-one correspondence is $a=b^2, a \in A, b \in B$, and the two sets are equivalent.
\end{boxexample}

Another good definition to know is when an infinite set is {\bf countable}, though we will not use it much in this class.

\begin{boxdefine}{Countable Set}{}
	An infinite set is {\bf countable} if it is equivalent to the natural numbers. 
\end{boxdefine}

We can also determine if one infinite set is a {\bf proper subset} of another infinite set. If each member of $A$ is paired with one and only one member in $B$, but not every member of $B$ is paried with one and only one member of $A$, than we say $A$ is a proper subset of $B$.

\begin{boxexample}{}{}
	If $A=\{2,3,4,5,\dots\}$ and $B=\{1,2,3,4,5,\dots\}$, than $A$ is a proper subset of $B$, because the one-to-one corresponce $a=b, a \in A, b \in B$
\end{boxexample}

\chapter{Logic I}

\begin{boxdefine}{Logical Statement}{}
A {\bf logical statement} is an expression that has a definite state of being either true or false.
\end{boxdefine}

Logic is the language of mathematics. We use logic to express (and prove!) ideas to another who speaks the language. When written, we aim to leave no doubt about our meaning. Of course, logic can also used to arrive at the correct results.

\begin{boxremark*}{}{}
	A {\bf paradox} is a statement that is both true and false, but can only exist within an inconsistant system of logic (and is not considered a statement within a consistem system of logic). For example, "This statement is false."
\end{boxremark*}

\begin{boxexample}{}{}
	Here are some examples of logical statements
	\begin{itemize}
		\item A triangle has three sides. (a true statement)
		\item $\sqrt 2$ is a rational number. (a false statement)
		\item $\pi + 2 \leq 5$. (a true statement)
	\end{itemize}
\end{boxexample}

\begin{boxexample}{}{}
	Here are some examples of non-logical statements
	\begin{itemize}
		\item Find the smallest natural number. (instruction, not statment)
		\item Does 3=9? (question, not statement)
		\item This statement is false. (cannot be true or false)
	\end{itemize}
\end{boxexample}

\section{Logical Connectors}

This is a list of the logical connectors. They work as you would use them in English.

\medskip
\begin{tabular}{r|c|c|l}
	\hline
	Connector & Symbol & Name & Example\\
	\hline
	Not & $\neg$ & Negative & $\neg \text{False} = \text{True}$\\
	And & $\land$ & Conjunction & $\text{True} \land \text{True} = \text{True} $\\
	Or & $\lor$ & Disjunction & $\text{True} \lor \text{False} = \text{True}$\\
	Implies (If...Then) & $\implies$ & Conditional & $\text{True} \implies \text{True} = \text{True}$\\
	Iff (If and only if) & $\iff$ & Biconditional & $\text{True} \iff \text{False} = \text{False}$\\
\hline
\end{tabular}
\medskip

\begin{boxnotation*}{}{}
	It is \emph{very} uncommon to use the set operations $\cup$ for OR and $\cap$ for AND, but the instructor sometimes uses these. The negative is sometimes written as "$\sim A$".
\end{boxnotation*}

Lets define how these resolve using truth tables. The first truth table we will look at is {\bf not}:

\medskip
\begin{tabular}{c|c}
	\hline
	\multicolumn{2}{c}{NOT Truth Table}\\
	\hline
	$A$ & $\neg A$\\
	\hline
	True & False\\
	False & True\\
	\hline
\end{tabular}
\medskip

\begin{boxexample}{}{}
	If $A=\text{True}$, what is $\neg A$?
	
	$\neg A=\text{False}$.
\end{boxexample}

Then we have the AND logical connector. Again, this works how you'd expect in english.

\medskip
\begin{tabular}{c|c|c}
	\hline
	\multicolumn{3}{c}{AND Truth Table}\\
	\hline
	$A$ & $B$ & $A \land B$\\
	\hline
	True & True & True\\
	False & True & False\\
	True & False & False\\
	False & False & False\\
	\hline
\end{tabular}
\medskip

\begin{boxexample}{}{}
	If $A=\text{True}$ and $B=\text{False}$, what is $A \land B$?
	
	$A \land B = \text{False}$.
\end{boxexample}

Then the OR logical connector. This one is also works how you'd expect it in english.

\medskip
\begin{tabular}{c|c|c}
	\hline
	\multicolumn{3}{c}{OR Truth Table}\\
	\hline
	$A$ & $B$ & $A \lor B$\\
	\hline
	True & True & True\\
	False & True & True\\
	True & False & True\\
	False & False & False\\
	\hline
\end{tabular}
\medskip

\begin{boxexample}{}{}
	If $A=\text{True}$ and $B=\text{False}$, what is $A \lor B$?
	
	$A \lor B = \text{True}$.
\end{boxexample}

Implies works a little differently. You can see implies also written as "if...then" statements. Pay attention to how this truth table works, and try to use it in English.

\medskip
\begin{tabular}{c|c|c}
	\hline
	\multicolumn{3}{c}{Implies Truth Table}\\
	\hline
	$A$ & $B$ & $A \implies B$\\
	\hline
	True & True & True\\
	False & True & True\\
	True & False & False\\
	False & False & True\\
	\hline
\end{tabular}
\medskip

I'm going to spend some extra time explaining this one with some more examples.

\begin{boxexample}{}{}
	Consider this example: "If it's raining, then I will bring an unbrella." That means the following statements are also true:
	\begin{itemize}
		\item It is raining, so I brought an unbrella.
		\item It is not raining, but I brought an unbrella.
		\item It is not raining, so I didn't being an unbrella.
	\end{itemize}
	But, the statement "It is raining, but I didn't bring an unbrella", means that I am lieing.
\end{boxexample}

\begin{boxexample}{}{}
	If $A=\text{True}$ and $B=\text{False}$, what is $A \implies B$?
	
	$A \implies B = \text{False}$.
\end{boxexample}

Finally, iff (or if any only if) works simularly to above, but it works both ways.

\medskip
\begin{tabular}{c|c|c}
	\hline
	\multicolumn{3}{c}{Iff Truth Table}\\
	\hline
	$A$ & $B$ & $A \iff B$\\
	\hline
	True & True & True\\
	False & True & False\\
	True & False & False\\
	False & False & True\\
	\hline
\end{tabular}
\medskip

I will use the same example as above to illistrate the difference.

\begin{boxexample}{}{}
	Consider this example: "I will bring an unbrella if and only if it is raining." That means the following statements are also true:
	\begin{itemize}
		\item It is raining, so I brought an unbrella.
		\item It is not raining, so I didn't being an unbrella.
	\end{itemize}
	But the following statements mean that I am lieing:
	\begin{itemize}
		\item It is raining, but I didn't bring an unbrella.
		\item It is not raining, but I brought an umbrella.
	\end{itemize}
\end{boxexample}

\begin{boxexample}{}{}
	If $A=\text{False}$ and $B=\text{True}$, what is $A \iff B$?
	
	$A \iff B = \text{False}$.
\end{boxexample}

{\bf Simple statements} are "single ideas." They often express compairison, like "is", "equals", "less than", etc. They are the \emph{nouns} of logic. {\bf Logical connectors} are the \emph{verbs} of logic, and connect simple statements togeather to form {\bf compound statements}

\begin{boxdefine}{Logical Connector, Simple \& Compound Statements}{}
	A {\bf simple statement} is a "single idea" statement that does not use a {\bf logical connector} like "and", "or", "not", "implies", and "iff".
	A {\bf compound statement} contains two or more simple statements connected by {\bf logical connectors}.
\end{boxdefine}

\begin{boxexample}{}{}
	An example of a {\bf simple statement} is: $a^2=49$. An example of a {\bf compound statement} is: if $a=7$ then $a^2=49$ (also: $a=7 \implies a^2=49$). 
\end{boxexample}

Two important definitions for creating logical statements are {\bf tautologies}, and {\bf contradictions}.

\begin{boxdefine}{Tautology and Contradictions}{}
	A {\bf tautology} is a statement that is always true. A {\bf contradiction} is a statement that is always false.
\end{boxdefine}
