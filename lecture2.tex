\chapter{Set Theory II}

\section{Venn Diagrams}

We can use {\bf Venn diagrams} to visualise operations between sets. For example, below the left circle represents the $A$ set, and the right circle reporesents the $B$ set. The coloured portion of the graphic represents the intersection (shared elements) between $A$ and $B$. So for $A \cap B$:

\begin{venndiagram2sets}[shade=skyblue,showframe=false]
	\fillACapB
\end{venndiagram2sets}

\myexample
{
	Suppose $A=\{a,b,c\}$ and $B=\{a,b,x,y\}$, then $A \cap B=\{a,b\}$ is visualised with:\\
	\begin{venndiagram2sets}[shade=skyblue,showframe=false,labelOnlyA={c},labelOnlyB={x,y},labelAB={a,b}]
		\fillACapB
	\end{venndiagram2sets}
}

This time we will visualise the union of two sets. Again, the coloured area represents everything withing the union. So for $A \cup B$:

\begin{venndiagram2sets}[shade=skyblue,showframe=false]
	\fillA \fillB
\end{venndiagram2sets}

\myexample
{
	Suppose $A=\{a,b,c\}$ and $B=\{a,b,x,y\}$, then $A \cup B=\{a,b,c,x,y\}$  is visualised with:\\
	\begin{venndiagram2sets}[shade=skyblue,showframe=false,labelOnlyA={c},labelOnlyB={x,y},labelAB={a,b}]
		\fillA \fillB
	\end{venndiagram2sets}
}

And now we look at the complement of the set. So everything within the \emph{universe of discourse} that is not in the set is coloured. So for $A^\complement$:

\begin{venndiagram2sets}[shade=skyblue,overlap=2.4cm,hgap=2.2cm,vgap=0.5cm,labelNotAB={\;\quad\qquad $\mathcal{U}$},labelB={}]
	\fillNotA
\end{venndiagram2sets}

\myexample
{
	Suppose $A=\{a,b,c\}$ and $\mathcal{U}=\{a,b,c,d,e,f,g,h\}$, then $A^\complement$ can be visualised with:\\
	\begin{venndiagram2sets}[shade=skyblue,overlap=2.4cm,hgap=2.2cm,vgap=0.5cm,labelNotAB={\;\quad\qquad $\mathcal{U}$ \: $d,e,f,g,h$},labelOnlyA={$a,b,c$},labelB={}]
		\fillNotA
	\end{venndiagram2sets}
}

And when we have two disjoint sets (two sets that do not share elements).

$A \cap B = \phi$\\
\begin{venndiagram2sets}[shade=skyblue,showframe=false,overlap=-.5cm]
\end{venndiagram2sets}

\myexample
{
	For example,\\
	\begin{venndiagram2sets}[shade=skyblue,showframe=false,overlap=-.5cm,labelOnlyA={a,b,c},labelOnlyB={1,2,3}]
	\end{venndiagram2sets}
}

$A-B$\\
\begin{venndiagram2sets}[shade=skyblue,showframe=false]
	\fillOnlyA
\end{venndiagram2sets}

\myexample
{
	For example,\\
	\begin{venndiagram2sets}[shade=skyblue,showframe=false,labelOnlyA={a,b,c,d},labelAB={a,b}]
		\fillOnlyA
	\end{venndiagram2sets}
}

$(A \cap B)^\complement$\\
\begin{venndiagram2sets}[shade=skyblue,labelNotAB={$\mathcal{U}$}]
	\fillNotAorNotB
\end{venndiagram2sets}

$(A \cap B)^\complement \cup B$\\
\begin{venndiagram2sets}[shade=skyblue,labelNotAB={$\mathcal{U}$}]
	\fillAll
\end{venndiagram2sets}

$(A \cap B) \cup (A \cup B^\complement)^\complement$\\
$A \cap B$\\
\begin{venndiagram2sets}[shade=skyblue,labelNotAB={$\mathcal{U}$}]
	\fillACapB
\end{venndiagram2sets}

$A \cup B^\complement$\\
\begin{venndiagram2sets}[shade=skyblue,labelNotAB={$\mathcal{U}$}]
	\fillA \fillNotAorB
\end{venndiagram2sets}

$(A \cup B^\complement)^\complement$\\
\begin{venndiagram2sets}[shade=skyblue,labelNotAB={$\mathcal{U}$}]
	\fillOnlyB
\end{venndiagram2sets}

$(A \cap B) \cup (A \cup B^\complement)^\complement = B$\\
\begin{venndiagram2sets}[shade=skyblue,labelNotAB={$\mathcal{U}$}]
	\fillB
\end{venndiagram2sets}
