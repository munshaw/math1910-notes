\chapter{Set Theory II}

We will use {\bf Venn diagrams} to visualise operations between sets. I think they are pretty self explanitory, so I will mostly go through a few examples here and move on. For example, below the left circle represents the $A$ set, and the right circle represents the $B$ set. The coloured portion of the graphic represents the intersection (shared elements) between $A$ and $B$. So for $A \cap B$:

\begin{venndiagram2sets}[shade=skyblue,showframe=false]
	\fillACapB
\end{venndiagram2sets}

\begin{boxexample}{}{}
	Suppose $A=\{a,b,c\}$ and $B=\{a,b,x,y\}$, then $A \cap B=\{a,b\}$ is visualised with:\\
	\begin{venndiagram2sets}[shade=skyblue,showframe=false,labelOnlyA={c},labelOnlyB={x,y},labelAB={a,b}]
		\fillACapB
	\end{venndiagram2sets}
\end{boxexample}

This time we will visualise the union of two sets. Again, the coloured area represents everything withing the union. So for $A \cup B$:

\begin{venndiagram2sets}[shade=skyblue,showframe=false]
	\fillA \fillB
\end{venndiagram2sets}

\begin{boxexample}{}{}
	Suppose $A=\{a,b,c\}$ and $B=\{a,b,x,y\}$, then $A \cup B=\{a,b,c,x,y\}$  is visualised with:\\
	\begin{venndiagram2sets}[shade=skyblue,showframe=false,labelOnlyA={c},labelOnlyB={x,y},labelAB={a,b}]
		\fillA \fillB
	\end{venndiagram2sets}
\end{boxexample}

And now we look at the complement of the set. So everything within the \emph{universe of discourse} that is not in the set is coloured. So for $A^\complement$:

\begin{venndiagram2sets}[shade=skyblue,overlap=2.4cm,hgap=2.2cm,vgap=0.5cm,labelNotAB={\;\quad\qquad $\mathcal{U}$},labelB={}]
	\fillNotA
\end{venndiagram2sets}

\begin{boxexample}{}{}
	Suppose $A=\{a,b,c\}$ and $\mathcal{U}=\{a,b,c,d,e,f,g,h\}$, then $A^\complement=\{d,e,f,g,h\}$ can be visualised with:\\
	\begin{venndiagram2sets}[shade=skyblue,overlap=2.4cm,hgap=2.2cm,vgap=0.5cm,labelNotAB={\;\quad\qquad $\mathcal{U}$ \: $d,e,f,g,h$},labelOnlyA={$a,b,c$},labelB={}]
		\fillNotA
	\end{venndiagram2sets}
\end{boxexample}

And when we have two disjoint sets (two sets that do not share elements). So for $A \cap B = \phi$:

\begin{venndiagram2sets}[shade=skyblue,showframe=false,overlap=-.5cm]
\end{venndiagram2sets}

\begin{boxexample}{}{}
	For example, given the two sets $A=\{a,b,c\}$, and $B=\{1,2,3\}$, we can see that they share no elements and are disjoint.\\
	\begin{venndiagram2sets}[shade=skyblue,showframe=false,overlap=-.5cm,labelOnlyA={a,b,c},labelOnlyB={1,2,3}]
	\end{venndiagram2sets}
\end{boxexample}

Yet again for set difference. $A-B$:

\begin{venndiagram2sets}[shade=skyblue,showframe=false]
	\fillOnlyA
\end{venndiagram2sets}

\begin{boxexample}{}{}
	For example, if $A=\{a,b,c,d\}$ and $B=\{c,d\}$, than $A-B=\{c,d\}$.\\
	\begin{venndiagram2sets}[shade=skyblue,showframe=false,labelOnlyA={c,d},labelAB={a,b}]
		\fillOnlyA
	\end{venndiagram2sets}
\end{boxexample}

\section{Applications for Venn Diagrams}

\begin{boxexample}{}{}
	Suppose we want to visualise $(A \cap B)^\complement \cup B$. Thats a little long so lets break it up. Recall what $A \cap B$ looks like.\\
	\begin{venndiagram2sets}[shade=skyblue,labelNotAB={$\mathcal{U}$}]
		\fillACapB
	\end{venndiagram2sets}\\
	Now, we'll invert that to get $(A \cap B)^\complement$:\\
	\begin{venndiagram2sets}[shade=skyblue,labelNotAB={$\mathcal{U}$}]
		\fillNotAorNotB
	\end{venndiagram2sets}\\
	And now lets add $B$. So, $(A \cap B)^\complement \cup B$ is:\\
	\begin{venndiagram2sets}[shade=skyblue,labelNotAB={$\mathcal{U}$}]
		\fillAll
	\end{venndiagram2sets}\\
	And we see that $(A \cap B)^\complement \cup B$ is acutally the universal set $\mathcal{U}$!
\end{boxexample}

\begin{boxexample}{}{}
	Let's try another one. $(A \cap B) \cup (A \cup B^\complement)^\complement$. Again, let's break this down. $A \cap B$ is simple enough.\\
	\begin{venndiagram2sets}[shade=skyblue,labelNotAB={$\mathcal{U}$}]
		\fillACapB
	\end{venndiagram2sets}\\
	The next one is $A \cup B^\complement$, a little bit more tricky. That can be read as the area of A, plus the area of not B. Think about it for a bit, its a logic puzzle.
	\begin{venndiagram2sets}[shade=skyblue,labelNotAB={$\mathcal{U}$}]
		\fillA \fillNotAorB
	\end{venndiagram2sets}\\
	We take the complement of above (inverting it) to get $(A \cup B^\complement)^\complement$:\\
	\begin{venndiagram2sets}[shade=skyblue,labelNotAB={$\mathcal{U}$}]
		\fillOnlyB
	\end{venndiagram2sets}\\
	And now we glue them together to get to get our union $(A \cap B) \cup (A \cup B^\complement)^\complement$:\\
	\begin{venndiagram2sets}[shade=skyblue,labelNotAB={$\mathcal{U}$}]
		\fillB
	\end{venndiagram2sets}\\
	From this digram, we can see there is another way of writting $(A \cap B) \cup (A \cup B^\complement)^\complement$. It's just the set $B$!
\end{boxexample}

\section{Applications for Sets}

\begin{boxexample}{}{}
	Suppose 150 children were surveyed. It was found that 35 played hockey, 71 played baseball, 30 played soccer, 10 played all three, 3 played only soccer, 17 played only hockey, 6 played only soccer and hockey, 48 played only baseball, and 53 played no sports.\\
	First, lets draw the Venn Diagram. Start by drawing the three sets.\\
	\begin{venndiagram3sets}[shade=skyblue,labelA={H},labelB={B},labelC={S},labelNotABC={$\mathcal{U}$}]
	\end{venndiagram3sets}\\
	Now, let's fill some of this in. Since 53 didn't play anything, they are outside of the three sets. 3 Played only soccer, 17 played only hockey and 48 played only baseball. 10 Played all three, and 6 played only soccer and hockey. So we have:\\
	\begin{venndiagram3sets}[shade=skyblue,labelA={H},labelB={B},labelC={S},labelNotABC={\quad $\mathcal{U} \quad 53$},labelOnlyA={17},labelOnlyB={48},labelOnlyC={3},labelABC={10},labelOnlyAC={6}]
	\end{venndiagram3sets}\\
	We'll fill in some more intersections now. How many played only hockey and baseball? There are 35 hockey players, take away 17 who played only hockey, take away 10 who played all three, and take away 6 who played hockey and soccer. So $35-17-10-6=2$. How many played only soccer and baseball? 71 baseball players, take away 48 who played baseball only, take away 10 who played all three, take away 2 who played hockey and baseball. So, $71-48-10-2=11$. The diagram looks like:\\
	\begin{venndiagram3sets}[shade=skyblue,labelA={H},labelB={B},labelC={S},labelNotABC={\quad $\mathcal{U} \quad 53$},labelOnlyA={17},labelOnlyB={48},labelOnlyC={3},labelABC={10},labelOnlyAB={2},labelOnlyBC={11},labelOnlyAC={6}]
	\end{venndiagram3sets}\\
	Now we can use this to answer some questions.
	\begin{enumerate}
		\item How many played soccer and baseball only? $11$
		\item How many played hockey and baseball only? $2$
		\item How many played soccer and baseball? $10+11=21$
		\item How many played soccer {\bf or} baseball but not hockey? $48+11+3=62$
		\item How many played exactly 2 games? $6+2+11=19$
		\item How many played exactly 1 game? $17+48+3=68$
		\item How many played only soccer? $3$
		\item How many played soccer {\bf and} baseball but not hockey? $3+48=51$
	\end{enumerate}
\end{boxexample}

\begin{boxexample}{}{}
	Fourty students who play board games was surveyed.\\
	\begin{tabular}{r|l}
		\hline
		Number of students & Game(s) played\\
		\hline
		18 & Chess\\
		20 & Scrabble\\
		27 & Carrom\\
		7  & Chess \& Scrabble\\
		12 & Scrabble \& Carrom\\
		4  & Chess \& Carrom \& Scrabble\\
		\hline
		\multicolumn{2}{c}{40 Total}\\
		\hline
	\end{tabular}\\
	We begin by drawing our Venn diagram. We are given the Chess \& Carrom \& Scrabble intersection as $4$. We can use that to find our Chess \& Scrabble only ($7-4=3$) and Scrabble \& Charrom only ($12-4=8$) intersections. Finally, we can calculate our scrabble-only section ($20-3-4-8=5$).\\
	\begin{venndiagram3sets}[shade=skyblue,showframe=false,labelA={Chess},labelB={Scrabble},labelC={Carrom},labelABC={4},labelOnlyAB={3},labelOnlyBC={8},labelOnlyB={5}]
	\end{venndiagram3sets}\\
	Now it gets tricky. Consider the following highlighted section ($\text{Chess} \cup \text{Carrom} - \text{Scrabble}$):\\
	\begin{venndiagram3sets}[shade=skyblue,showframe=false,labelA={Chess},labelB={Scrabble},labelC={Carrom},labelABC={4},labelOnlyAB={3},labelOnlyBC={8},labelOnlyB={5}]
		\fillOnlyA \fillOnlyC \fillACapCNotB
	\end{venndiagram3sets}\\
	There must be $40-20=20$ students in there. We know that $27-8-4=15$ of them play Carrom, and that $18-4-3=11$ of them play chess. So the Chess \& Carrom only intersection must be $15+11-20=6$. Thus the chess-only section is $11-6=5$, and the carrom-only section is $15-6=9$.\\
	\begin{venndiagram3sets}[shade=skyblue,showframe=false,labelA={Chess},labelB={Scrabble},labelC={Carrom},labelABC={4},labelOnlyAB={3},labelOnlyBC={8},labelOnlyB={5},labelOnlyAC={6},labelOnlyA={5},labelOnlyC={9}]
	\end{venndiagram3sets}\\
	The number of people who play both Chess \& Carrom is $6+4=10$, and the number of people who play Chess and Carrom but not Scrabble is $6$.
\end{boxexample}
